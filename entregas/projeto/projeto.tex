\documentclass[12pt,letterpaper]{article}
\usepackage[utf8]{inputenc}
\usepackage{amsmath,amsthm,amsfonts,amssymb,amscd}
\usepackage[table]{xcolor}
\usepackage[margin=2.5cm]{geometry}
\usepackage{ragged2e}
\usepackage{graphicx}
\usepackage{multicol}
\usepackage{hyperref}
%\usepackage[brazil]{babel}
\newlength{\tabcont}
\setlength{\parindent}{0.0in}
\setlength{\parskip}{0.05in}

\begin{document}
	
	\begin{center}
		\LARGE \bf
		Projeto para MAC0214 - Atividade Curricular em Cultura e Extensão
	\end{center}
	
	\large \textbf{Nome}: Luís Felipe de Melo Costa Silva \\
	\textbf{Número USP}: 9297961 \\
	\large \textbf{Docente Orientador}: José Coelho de Pina Junior
	
	Meu projeto para a disciplina MAC0214 consiste em realizar atividades que divulgam o curso de Bacharelado em Ciência da Computação e participar da organização de dois grandes eventos apreciados pela comunidade desse curso: o \textit{Encontro do BCC} e o \textit{HackathonUSP}.
	
	\section{Divulgação do Curso}
	
	\subsection{Palestra do Ensino Médio}
	
	Há pelo menos três anos damos uma palestra para alunos do Ensino Médio e vestibulandos, como um subevento do Encontro do BCC. É uma apresentação curta, que consiste em dar uma visão geral do curso, explicando as diferenças entre os cursos de Ciência da Computação (do IME), Sistemas da Informação (da EACH) e Engenharia da Computação (da POLI). Depois disso, temos uma sessão de perguntas e respostas e uma roda de conversa. 
	
	Geralmente, um aluno do segundo ano oferece essa palestra, como eu fiz no ano passado. Ela é divulgada em algumas escolas e alguns cursinhos da cidade de São Paulo, com cartazes e um evento do \textit{Facebook}.
	
	\textbf{O que farei:} Estarei presente nesta palestra durante toda a sua duração. Ela vai ocorrer no \underline{dia 18/08, às 15h}, no Auditório Jacy Monteiro
	
	\subsection{Feira de Profissões da USP}
	
	Todos os anos a USP organiza essa feira. É organizada pela Pró-Reitoria de Cultura e Extensão Universitária da USP, e realizada tanto na capital quanto nos campi do interior. O público-alvo é o mesmo da Palestra do Ensino Médio.
	
	A feira consiste em varios \textit{stands}, creio que um por instituto. Cada um contém professores e monitores, que fornecem informações sobre cada curso e respondem às dúvidas dos participantes.
	
	\textbf{O que farei:} Eu me candidatei para monitor para a feira desse ano, para representar o BCC. Fui selecionado, e irei no \underline{dia 26/08, sábado}. Esse é o último dia da feira, e os monitores ajudam a desmontar o stand.
	
	\section{Encontro do BCC}
	
	O Encontro já é um evento tradicional do nosso curso, realizado desde 2009. Participo da organização desde quando entrei, no ano de 2015.	Ele é feito pelos alunos e supervisionado pelos professores. É dirigido para todos os que têm interesse pela computação.
	
	Basicamente, é um ciclo de palestras com temas escolhidos pelos alunos, para mostrar o que o curso pode proporcionar. Além disso, acontecem subeventos relacionados. Esse ano, ele acontecerá \underline{entre os dias 21/08 e 25/08}, no IME. 
	
	Como pode-se ver, ele acontece no início do segundo semestre, mas a organização está trabalhando desde o início do primeiro semestre. Existem várias equipes, que são:
	
	\begin{itemize}
		\item Palestras: chama os palestrantes para o evento.
		\item Feira de Livros: entra em contato com editoras e reservam o material necessário para que elas possam vender seus livros.
		\item Patrocínios: conversa com empresas que podem ter interesse em fornecer verba para o evento.
		\item Divulgação: utiliza um site, cartazes e redes sociais para espalhar o evento.
		\item Ofícios: cuida de toda a burocracia com o instituto, desde reservas até a captação de verba institucional.
		\item Comida: faz o orçamento dos coffee-breaks e escolhe o lugar com o melhor custo-benefício para fornecê-los.
	\end{itemize}
	
	\textbf{O que farei:} Esse ano, estou participando do time de Palestras e cuido bastante dos Patrocínios. Além disso, ajudo em menor escala do time de Ofícios. Também estou cuidando de parte da administração interna da organização (reserva de salas para reunião, lembrete de organização, auxílio com contatos, entre outros). Pretendo estar presente durante toda a semana do Encontro, cuidando de meus palestrantes, ajudando com os subeventos, resolvendo imprevistos, entre outras atividades, tudo para que o evento mantenha sua qualidade.
	
	\section{HackathonUSP}
	
	Participo do IME Workshop, que traz palestras e oficinas de interesse dos estudantes. Esse ano, estamos organizando um \textit{Hackathon} em parceria com o NEU (Núcleo de Empreendedorismo da USP) e a Pró-Reitoria de Pesquisa da USP.
	
	Um \textit{Hackathon} é um evento em que os participantes têm que desenvolver (programar) um projeto relacionado a um certo tema num período de 24 horas. 
	
	Esse ano, o HackathonUSP será nos dias \underline{19 e 20 de agosto, no CCSL}. Assim como o Encontro do BCC, a organização está acontecendo desde antes disso. 
	
	Estamos fazendo reuniões com o NEU, e basicamente, estamos cuidando da parte de ofícios do IME enquando o NEU cuida inteiramente dos patrocínios.
	
	Comparado com o Encontro do BCC, o HackathonUSP tem uma organização mais simples, e o maior problema é a sua realização, já que envolve uma virada de noite. 
	
	\textbf{O que farei:} Além de ter ajudado na organização, ficarei durante todo o evento, ajudando na execução de tarefas relacionadas ao evento e à comida e resolvendo potenciais imprevistos.
	
	\section{Horas estimadas de trabalho}
	
	\subsection{Divulgação do Curso}
	
	\subsubsection{Palestra do Ensino Médio}
	
	A palestra terá uma duração estimada de \underline{2 horas}, e estarei lá.
	
	\subsubsection{Feira de Profissões da USP}
	
	Ficaremos 9 horas na feira e mais umas 2 horas desmontando o stand e arrumando os materiais, um total de \underline{11 horas}.
	
	\subsection{Encontro do BCC}
	
	Contando o tempo já gasto na organização e somando com o tempo que será usado no futuro, teremos um total de \underline{50 horas}, aproximadamente.
	\begin{itemize}
		\item 30 horas, na semana do evento;
		\item 10 horas, em reuniões;
		\item 10 horas, com os trabalhos internos (contato com patrocinadores e palestrantes, ofícios do IME, entre outros).
	\end{itemize}
	
	\subsection{HackathonUSP}
	
	Colocando o tempo que já utilizamos mais o tempo do evento, o evento tomará \underline{41 horas}.
	\begin{itemize}
		\item 30 horas, no evento;
		\item 6 horas de reuniões; 
		\item 5 horas de trabalho extra (ofícios, reservas, arrumar o local, entre outros).
	\end{itemize}
	
	No total, dedicarei \underline{104 horas} a essas atividades, aproximadamente.
	
	\section{Acompanhamentos}
	
	A página para acompanhamento do meu trabalho na disciplina será \url{https://lsflp.github.io/MAC0214}. Nela estarão:
	
	\begin{itemize}
		\item Atas das Reuniões do Encontro do BCC e do HackathonUSP.
		\item Evidências do meu trabalho (principalmente, cópias de e-mails).
		\item Relatos que descrevem os eventos.
		\item Fotos das atividades.
	\end{itemize}

			 
\end{document}