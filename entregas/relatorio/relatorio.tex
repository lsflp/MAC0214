\documentclass[12pt,letterpaper]{article}
\usepackage[utf8]{inputenc}
\usepackage{amsmath,amsthm,amsfonts,amssymb,amscd}
\usepackage[table]{xcolor}
\usepackage[margin=2.5cm]{geometry}
\usepackage{ragged2e}
\usepackage{graphicx}
\usepackage{multicol}
\usepackage{hyperref}
%\usepackage[brazil]{babel}
\newlength{\tabcont}
\setlength{\parindent}{0.0in}
\setlength{\parskip}{0.05in}

\begin{document}

	\begin{center}
		\Large \bf
		Relatório sobre MAC0214 - Atividade Curricular em Cultura e Extensão
	\end{center}
	
	\textbf{Nome}: Luís Felipe de Melo Costa Silva \\
	\textbf{Número USP}: 9297961
	
	\section*{Introdução}
	Para essa disciplina eu havia proposto participar da organização do Encontro do BCC (e seu subevento Palestra do Ensino Médio) e do HackathonUSP. Além disso, prometi estar na Feira de Profissões da USP, como expositor do curso.
	
	\section{Encontro do BCC}
	O Encontro do BCC é um evento que acontece todos os anos desde 2009. Quem organiza são os próprios alunos do curso, e o público-alvo principal são os próprios alunos, além de quem se interessa por Computação de um modo geral.
	
	O Encontro chegou à sua IX edição esse ano. Tenho participado desde quando entrei, em 2015, então, já tenho uma noção do que tem que ser feito e de como o evento tem que parecer. 
	
	Basicamente, o que temos que preparar é:
	
	\begin{itemize}
		\begin{multicols}{2}
			\item Ciclo de Palestras;
			\item Conversa com os Professores do MAC;
			\item Ofícios;
			\item Divulgação;
			\item Patrocínios;
			\item Comida;
			\item Feira de Livros;
			\item Palestra do Ensino Médio.
		\end{multicols}
	\end{itemize}

	Abaixo, segue em detalhes o que é cada um desses itens e qual foi a minha contribuição em cada um deles.
	
	\subsection{Ciclo de Palestras}
	
	\subsection{Conversa com os Professores e Entrega do PIPA}
	
	\subsection{Ofícios}
	
	\subsection{Divulgação}
	
	\subsection{Patrocínios}
	
	\subsection{Comida}
	
	\subsection{Palestra do Ensino Médio}
	
	\subsection{A semana}
	
	\section{HackathonUSP}
	
	\subsection{Preparação}
	
	\subsection{O evento}
	
	\section{Feira de Profissões}
	
	\section{Considerações finais}
	
	\subsection{Encontro do BCC}
	
	\subsection{HackathonUSP}
	
	\subsection{Feira de Profissões}
			 
\end{document}